\section{\label{sec:Gruppen}Gruppen}

  \textbf{Fundamentales}:
  \begin{items}
    \item \underline{Verknüpfung}: $\star : M \times M \rightarrow M$
    \item \underline{Gruppe}: $(M,\star)$ mit:
      \begin{enumeration}
        \item $\star$ assoziativ
        \item neutrales Element $e$ ($\forall m \in M: m \star e = e \star m = m$)
        \item inverse Elemente $m^{-1}$ ($\forall m \in M: m \star m^{-1} = m^{-1} \star m = e$)
      \end{enumeration}
    \item \underline{abelsche Gruppe}: $\star$ kommutativ (auch kommutative Gruppe)
  \end{items}

  \textbf{Untergruppe}:
  \begin{items}
    \item $(H \subseteq G, \circ)$ Untergruppe von $(G, \star)$, wenn
    \begin{enumeration}
      \item $(H, \circ)$ Gruppe
      \item $\forall h_1, h_2 \in H: h_1 \circ h_2 = h_1 \star h_2$
    \end{enumeration}
    \item \underline{Untergruppenkriterium}: $H \neq \varnothing \wedge \forall h_1, h_2 \in H: h_1 \star h_2^{-1} \in H$
    \item \underline{Gruppendurchschnitt}: $G$ Gruppe, $I \neq \varnothing, \forall i \in I: U_i$ Untergruppe von $G$ $\leadsto \bigcap_{i \in I} U_i$ Untergruppe von $G$
    \item \underline{Gruppenerzeugnis}: $M \subseteq G, I=\{ X: X \text{ Untergruppe von } G \wedge X \text{ enthält } M\}$, $\langle M \rangle=\bigcap_{X \in I} X$ Gruppenerzeugnis von $M$
    \item \underline{zyklische Gruppe}: $\exists a \in G: G = \langle a \rangle$
    \item \underline{Ordnung}:
    \begin{enumeration}
      \item Gruppe: Kardinalität von $G$
      \item $g \in G$: $|\langle g \rangle |$
    \end{enumeration}
    \item $H$ Untergruppe von $G \Rightarrow |H|$ teilt $|G|$
  \end{items}

  \textbf{Gruppenhomomorphismus}:
  \begin{items}
    \item $=f:G \rightarrow H \ \forall x,y \in G: f(x \star y) = f(x) \circ f(y)$
    \item Eigenschaften:
    \begin{enumeration}
      \item $f(e_G) = e_H$
      \item $\forall g \in G: f(g)^{-1} = f(g^{-1})$
      \item $f^{-1}(\{ e_H \} )$ Untergruppe von $G$
      \item $f$ injektiv $\Leftrightarrow f^{-1}(\{ e_H \} ) = \{ e_g \}$
    \end{enumeration}
    \item \underline{$\text{Hom}(G,H)$}: Menge der Gruppenhomomorphismen von $G$ nach $H$
    \item \underline{Kern}: $=f^{-1}(\{ e_H \})$
    \item \underline{Endomorphismus}: $f \in \text{Hom}(G,G) \Leftrightarrow f \in \text{End}(G)$
    \item \underline{Isomorphismus}: $f \in \text{Hom}(G,H) \wedge f$ bijektiv $\leadsto f \in \text{Iso}(G,H)$ ($\text{Iso}(G,H) \neq \varnothing \Rightarrow G,H$ isomorph)
    \item \underline{Automorphismus}: $f \in \text{End}(G) \wedge f$ bijektiv $\leadsto f \in \text{Aut}(G)$
  \end{items}

  \textbf{Symmetrische Gruppe}:
  \begin{items}
    \item $D \text{ Menge}, M := \{ f \in \text{Abb}(D,D): f \text{ bijektiv} \}$ Gruppe mit Komposition $\leadsto$ symmetrische Gruppe $\text{Sym}_D = (M, \circ)$
    \item $D=\{ 1,\dots ,n \} \Rightarrow \text{Sym}_D =: S_n$ Permutationen der ersten $n$ Zahlen aus $\N$, $|S_n|=n!$
    \item \underline{$d$-Zykel}: $d \leq n$ Elemente aus $S_n$ werden im Kreis getauscht
    \item \underline{Transposition}: Zwei Elemente werden vertauscht ($2$-Zykel)
    \item \underline{Zerlegung}: Permutation zerlegbar in Transpositionen (durch Komposition): $\sigma = \tau_1 \circ \cdots \circ \tau_k$
    \item \underline{Signum}:
    \begin{enumeration}
      \item Transposition: $\text{sgn}(\tau)=-1$
      \item Permutation: $\sigma = \tau_1 \circ \cdots \circ \tau_k \leadsto \text{sgn}(\sigma)={(-1)}^{k}$
    \end{enumeration}
  \end{items}